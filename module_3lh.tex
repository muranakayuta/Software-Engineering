\documentclass[a4j]{jarticle}

\usepackage[dvipdfmx]{graphicx}
\usepackage{listings}

\begin{document}
\section{モジュール仕様}
各モジュールの仕様は以下の通りです.「」は画面のボタン名を表しています.
\subsection{complete.html.erb}
          【名称】\\
            タブレット端末 完了画面\\
           【概要】\\
            水量選択処理または乾燥量選択処理の完了画面を表示します.\\
           【処理フロー】\\
            ・「閉じる」を選択すると,select.html.erbを呼び出します\\
\subsection{h\_main.html.erb}
           【名称】\\
            管理者メインメニュー画面\\
           【概要】\\
            管理者用のメインメニュー画面を表示します.紛失物の一覧を見ることができ,新規登録や編集を行うことができます.また,「ログアウト」,「通知」を選択することもできます.\\
           【処理フロー】\\
            ・「新規登録」を選択すると,h\_registration.html.erbを呼び出します.\\
            ・「一覧・編集」を選択すると,h\_rer\_search.html.erbを呼び出します.\\
            ・「通知」を選択すると,g\_notification.html.erb が呼び出されます.\\
            ・「ログアウト」を選択すると確認のポップアップが表示され,「はい」を選択するとg\_login.html.erbが呼び出され,「いいえ」を選択するとポップアップが消えます.\\
\subsection{h\_reg\_search.html.erb}
           【名称】\\
            管理者検索画面\\
           【概要】\\
            管理者用の紛失物検索画面を表示します.検索したい紛失物の種類や色をチェックボックスで選択し,検索することができます.また,「ログアウト」,「通知」を選択することができます.\\
           【処理フロー】\\
            ・紛失物情報をチェックボックスで選択し,「検索」を選択すると,h\_reg\_search.rbを呼び出します.\\
            ・「通知」を選択すると,g\_notification.html.erb が呼び出されます.\\
            ・「ログアウト」を選択すると確認のポップアップが表示され,「はい」を選択するとg\_login.html.erbが呼び出され,「いいえ」を選択するとポップアップが消えます.\\
\subsection{h\_reg\_search.rb}
           【名称】\\
            管理者紛失物検索処理\\
           【概要】\\
            管理者用の紛失物検索処理を行います.\\
           【処理フロー】\\
            ・データベースに対して選択された項目を元に検索を行い,h\_reg\_result.html.erbを呼び出します.\\
\subsection{h\_reg\_result.html.erb}
           【名称】\\
            管理者検索結果画面\\
           【概要】\\
            管理者用の紛失物検索結果画面を表示します.表示された画像は選択することができます.また,「ログアウト」,「通知」を選択することができます.\\
           【処理フロー】\\
            ・表示された画像を選択すると,h\_loss\_read.rbを呼び出します.\\
            ・「通知」を選択すると,g\_notification.html.erb が呼び出されます.\\
            ・「ログアウト」を選択すると確認のポップアップが表示され,「はい」を選択するとg\_login.html.erbが呼び出され,「いいえ」を選択するとポップアップが消えます.\\
\end{document}
